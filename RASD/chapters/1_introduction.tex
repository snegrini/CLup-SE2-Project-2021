\chapter{Introduction}

\section{Purpose}
The purpose of this document is to present a detailed description of Customers Line-up (CLup).
It provides functional and non-functional requirements for the development of the system, including use cases, features, user interaction and system constraints.

This document is addressed to the developers who have to implement the requirements and could be used as an agreement between the customer and the contractors.

\subsection{Goals}\label{intro:goals}
The ambition is that the adoption of these requirements will avoid having crowds inside facilities such as supermarkets and relieve the consequent long lines forming outside. This matter is particularly relevant during a pandemic emergency, like the one that is going on in the present day.\newline
Below are presented the goals of CLup. Further description will be discussed in section \clupref{desc:prodFunc}.

\begin{enumerate}[label=\textbf{G.\arabic*}]
	\item \itemtext{goal:avoidQueue}{Avoid the creation of physical queues outside stores.}
	\begin{enumerate}[label*=\textbf{.\arabic*}, leftmargin=+.5in]
		\item \itemtext{goal:custHazardSit}{Allow customers to avoid the creation of hazardous situations.}
        \item \itemtext{goal:storeHazardSit}{Allow supermarkets to avoid the creation of hazardous situations.}
		\item \itemtext{goal:shortenTime}{Shorten the amount of time a customer is in queue.}
		\item \itemtext{goal:arriveOnTime}{Allow customers to arrive at stores right on time.}
	\end{enumerate}

	\item \itemtext{goal:otherTasks}{Allow customers to perform other tasks while they are queued up.}

    \item \itemtext{goal:easyExp}{Grant customers an overall experience as easy as possible.}

	\item \itemtext{goal:enjoyService}{Allow customers, even the ones who don't have access to technology, to enjoy the service.}

	\item \itemtext{goal:monitorAccess}{Allow supermarkets to monitor access to stores in a better way.}

	\item \itemtext{goal:knowInAdvance}{Allow supermarkets to know in advance how many people are coming to stores.}

    \item \itemtext{goal:limitNumber}{Allow supermarkets to limit the number of access to stores.}
\end{enumerate}

\section{Scope}
Customer Line-up (CLup) is an \textit{easy-to-use} application which aims to settle for various queuing problems faced by supermarkets and their customers.\newline
On the one side, it allows store managers to regulate the influx of people in the building and, on the other side, it saves people from having to line up and stand outside of stores for hours on end.

Customers can enter a queue in \textit{real-time} by taking a ticket via different channels such as Self Service Ticketing Kiosk and Mobile App. During this process, the user is given an estimation of the waiting time and the \textit{leave-at-time} (i.e. the time they need to depart from their current position to reach the store). This ticket comprehends a queue number, which identify user's position in the queue, and a QR code, which is used for the ticket validation.\newline
The validation process will be performed by a store employee using a dedicated application.

The system offers store managers a way to monitor the customers flow and check the journey map of all the clients inside the store at a given time.

The platform also support an \textit{advanced functionality} where a customer can "book a visit" to the store by indicating the approximate expected duration of the trip and the main categories of items they intend to buy. For long-term customers, it suggests a time inferred by the system based on an analysis of the previous visits.

This application works as a digital counterpart to the common situation where people who are in line for a service retrieve a number that gives their position in the queue. The \textit{legacy system} will be completely superseded by the application. Indeed, its effectiveness is strictly bound to the number of users who use it.

\subsection{World Phenomena}
\begin{enumerate}[label=\textbf{WP.\arabic*}]
	\item Customers choose which store to go to.
	\item Customers approach the chosen store.
	\item Supermarkets restrict access to their stores.
	\item Supermarkets monitor influx of people in the building.
	\item Customers line up outside the store.
	\item Customers wait their turn.
\end{enumerate}

\subsection{Shared Phenomena}
\begin{enumerate}[label=\textbf{SP.\arabic*}]
	\item Customers choose which store to go to.
	\item Ticket is retrieved from the customer remotely.
	\item Ticket is retrieved from the customer locally at the store.
	\item Customers can book a time-slot at the store.
	\item Customers check the queue status of the store.
	\item Ticket are validated by the system.
	\item Supermarkets monitor store data using the system.
\end{enumerate}

\section{Glossary}
\subsection{Definitions}
\begin{center}
	\begin{tabular}{@{}p{0.25\linewidth} p{0.71\linewidth}@{}}
		\toprule
		\textbf{Term} & \textbf{Definition}\\
		\midrule
		Supermarkets & Used in this document to mean supermarket managers.\\
		Stores & Used in this document to mean the physical building of the supermarket.\\
		Customers & Supermarket customers which are also the users of CLup.\\
		Employees & Used in this document to mean both entrance-staff and cashiers.\\
		Store Pass & General term that comprehends both tickets and bookings.\\
		Ticket & Pass generated from the system which is comprehensive of the Queue number and QR code.\\
		Booking, Reservation & Pass generated from the system as a result of the reservation process.\\
		Queue number & Identify user's position in the queue.\\
		QR code & Type of matrix barcode, used by the system for the ticket validation.\\
		System & Totality of the hardware/software applications that contribute to provide the service concerned. Also referred as CLup, Application, Platform.\\
		Legacy System & Used in this document to mean the physical ticketing system where you retrieve a ticket from a stand.\\
		\bottomrule
	\end{tabular}
\end{center}

\subsection{Acronyms}
\begin{center}
	\begin{tabular}{@{}p{0.25\linewidth} p{0.71\linewidth}@{}}
		\toprule
		\textbf{Acronyms} & \textbf{Term}\\
		\midrule
		CLup & Customers Line-up\\
		QR & Quick Response\\
		GPS & Global Positioning System\\
		UI & User Interface\\
		CAPTCHA & Completely Automated Public Turing test to tell Computers and Humans Apart\\
		\bottomrule
	\end{tabular}
\end{center}

\subsection{Abbreviations}
\begin{center}
	\begin{tabular}{@{}p{0.25\linewidth} p{0.71\linewidth}@{}}
		\toprule
		\textbf{Abbreviations} & \textbf{Term}\\
		\midrule
		e.g. & Exempli gratia\\
		i.e. & Id est\\
		w.r.t. & With reference to\\
		G & Goal\\
		WP & World Phenomena\\
		SP & Shared Phenomena\\
        R & Requirement\\
		\bottomrule
	\end{tabular}
\end{center}

\section{Reference documents}
\begin{itemize}
	\item Project assignment specification document.
	\item ISO/IEC/IEEE 29148 - Systems and software engineering.
	\item Course slides on beep.
\end{itemize}

\section{Document Structure}
This document is presented as it follows:
\begin{enumerate}
	\item \textbf{Introduction}: contains a preamble of the given problem and proposes, in a simple way, the system and its goals as solution.

	\item \textbf{Overall Description}: gives a general description of the system, focusing on its functions and constraints. Moreover, it provides the domain assumptions of the analysed world.

	\item \textbf{Specific Requirements}: explains in detail the functional and non functional requirements. It lists the possible interactions with the system in the form of scenarios, use cases and sequence diagrams.

	\item \textbf{Formal Analysis Using Alloy}: contains the Alloy model of some critical aspects of the system and an example of the generated world.

	\item \textbf{Effort Spent}: keeps track of the time spent to complete this document. The first table defines the hours spent	as a team for taking the most important decisions, the seconds contain the individual hours.
\end{enumerate}
