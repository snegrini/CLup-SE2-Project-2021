\chapter{Adopted Development Frameworks}
In addition to what has already been said in chapter 5 of the DD, additional libraries and frameworks have been adopted such as:
\begin{itemize}
	\item \textbf{EclipseLink}: provides an extensible framework that allows Java developers to interact with various data services, including databases, web services, Object XML mapping, and Enterprise Information Systems.
	
	\item \textbf{Maven}: is a build automation tool used primarily for Java projects.
	
	\item \textbf{Jackson}: is a high-performance JSON processor for Java used to build up messages between the applications and server.
	
	\item \textbf{Spring Security BCryptPasswordEncoder}: a standard when it comes to password hashing.
	
	\item \textbf{QR library zxing}: a Java library which allows to display QR codes inside web pages.
	
	\item \textbf{Bootstrap}: along with his template SBAdmin 2, it has been adopted for the front-end of the web application.
	
	\item \textbf{jQuery}: JavaScript library to manage AJAX requests and a Bootstrap dependency.
\end{itemize}

\section{Programming languages}

\subsection{Java}
 The Java Programming Language is a general-purpose, concurrent, strongly typed, class-based object-oriented language. It has been used for the server-side development along with Java EE: a set of specifications for enterprise features such as distributed computing and web services. Below are listed some of its advantages and disadvantages:
 
\textbf{Advantages}
\begin{itemize}
	\item it is object oriented: in Java, everything is an Object. Java can be easily extended since it is based on the Object model.
	
	\item it is platform independent: Java is compiled into platform independent byte code. This byte code is interpreted by the Virtual Machine (JVM) on whichever platform it is being run on.
	
	\item it is robust and mature: Java makes an effort to eliminate error prone situations by emphasizing mainly on compile time error checking and runtime checking.
\end{itemize}

\textbf{Disadvantages}
\begin{itemize}
	\item Java has high memory and processing requirements. Therefore, hardware cost increases;
	\item it does not provide support for low-level programming constructs like pointers;
	\item you don't have any control over garbage collection as Java does not offer functions like delete(), free().	
\end{itemize}


\subsection{Dart}
Dart is a client-optimized language for fast apps on any platform. It has been used as language of the client-side system because of its support of the Flutter framework. Below are listed some of its advantages and disadvantages:

\textbf{Advantages}
\begin{itemize}
	\item it is object oriented;
	\item it has a fast VM: programs written in Dart tend to run faster than programs created, for example, in JavaScript;
	\item it has robust core libraries, baked into the SDK;
	\item it has a built-in package manager;
	\item it has useful features like mixins, implicit interfaces, lexical closures, named constructors, one-line functions.
\end{itemize}

\textbf{Disadvantages}
\begin{itemize}
	\item it is not a mature language as Java;
	\item its package collection is not big as competition.	
\end{itemize}
