\chapter{Structure of the Source Code}

\section{Server side}
The server side is divided into two main modules: \textbf{CLupEJB} and \textbf{CLupWeb}.

\subsection{CLupEJB}
This module encapsulates the business logic of the application. An EJB web container provides a runtime environment for web related software components. Inside, it is divided into:
\begin{itemize}
	\item \textbf{Entities}: these are POJOs representing data that can be persisted to the database. Each entity represents a table stored in the database;
	\item \textbf{Enums}: custom enum types for specifying predefined constants;
	\item \textbf{Exceptions}: custom exception types for specifying errors;
	\item \textbf{Messages}: custom classes for sending data in a known format to the client;
	\item \textbf{Services}: Enterprise Java Bean, the core components of this module which also interacts with the EntityManager to persist data on the database;
	\item \textbf{Util}: other utility classes such as the TokenManager to handle the token authentication without credentials for the customers app.
\end{itemize}

\subsection{CLupWeb}
This module contains Java servlets that respond to web HTTP requests and qualify as a server-side servlet web API. Inside it is divided into:
\begin{itemize}
	\item \textbf{Controllers}: contains all the web servlets which serve different pages and performs calls on the injected EJB module;
	\item \textbf{Filters}: contains all the web filters used to check user authentication and permissions.
\end{itemize}
Moreover it includes all the static content like CSS, JS and HTML web pages which will be processed by the template engine Thymeleaf and served to the client web browser.


\section{Client side}
Both the client applications are made with flutter. In flutter, all the logic resides in the \textit{lib} folder. In the \textit{Android} and \textit{iOS} folders there are the respective platform files.

Below is presented the code structure of each app.
\subsection{Customer App}
In the lib folder there are:
\begin{itemize}
	\item \textbf{Enum}: contains the enums.
	\item \textbf{Model}: contains the model classes (such as Ticket, Store etc.).
	\item \textbf{Util}: contains the utility classes (such as the API Manager).
	\item \textbf{Views}: contains all the pages of the app.
\end{itemize}
\subsection{Store App}
In the lib folder there are:
\begin{itemize}
	\item \textbf{Util}: contains the utility classes (such as the API Manager).
	\item \textbf{Views}: contains all the pages of the app.
\end{itemize}