\chapter{Introduction}

\section{Purpose}
This document is the Implementation and Testing Document for the Customers Line-Up system. Here will be listed the implemented requirements, the used frameworks and the testing done

\section{Scope}
Its aim is to define and report about the software implementation and test procedures defined for all the released software. Precisely, it verifies that the CLup Platform as a whole satisfies its functional requirements. To this end, we define a set of tests covering the different uses that have been defined in the RASD and DD. The main goal of these tests  is  to  specify  for  each  tested  demonstrator scenario a series of actions leading to an expected result.

\section{Glossary}
\subsection{Definitions}
\begin{center}
	\begin{tabular}{@{}p{0.25\linewidth} p{0.71\linewidth}@{}}
		\toprule
		\textbf{Term} & \textbf{Definition}\\
		\midrule
		Customers &  Identifies the store customers.\\
		Employees & Used in this document to mean both entrance-staff and cashiers.\\
		Store Pass & General term that comprehends both tickets and bookings.\\
		Ticket & Pass generated from the system which is comprehensive of the Queue number and QR code.\\
		Booking, Reservation & Pass generated from the system as a result of the reservation process.\\
		Queue number & Identify user's position in the queue.\\
		QR code & Type of matrix barcode, used by the system for the ticket validation.\\
		System & Totality of the hardware/software applications that contribute to provide the service concerned. Also referred as CLup, Application, Platform.\\
		\bottomrule
	\end{tabular}
\end{center}

\subsection{Acronyms}
\begin{center}
	\begin{tabular}{@{}p{0.25\linewidth} p{0.71\linewidth}@{}}
		\toprule
		\textbf{Acronyms} & \textbf{Term}\\
		\midrule
		CLup & Customers Line-up\\
		RASD & Requirements Analysis and Specification Document\\
		DD & Design Document\\
		QR & Quick Response\\
		GPS & Global Positioning System\\
		UI & User Interface\\
		API & Application Programming Interface\\
		OS & Operating System\\
		HTTPS & HyperText Transfer Protocol Secure\\
		JSON & JavaScript Object Notation\\	
		DB & DataBase\\
		DBMS & DataBase Management System\\
		\bottomrule
	\end{tabular}
\end{center}

\subsection{Abbreviations}
\begin{center}
	\begin{tabular}{@{}p{0.25\linewidth} p{0.71\linewidth}@{}}
		\toprule
		\textbf{Abbreviations} & \textbf{Term}\\
		\midrule
		e.g. & Exempli gratia\\
		i.e. & Id est\\
		R & Requirement\\		
		\bottomrule
	\end{tabular}
\end{center}

\section{Reference documents}
\begin{itemize}
	\item Project assignment specification document.
    \item CLup, Requirements Analysis and Specification Document.
    \item CLup, Design Document.
	\item Course slides on beep.
\end{itemize}

\section{Document structure}
This document is presented as it follows:
\begin{enumerate}
	\item \textbf{Introduction}: the purpose of this document.

	\item \textbf{Implemented Requirements}: the requirements and functions that are actually implemented in the software.

	\item \textbf{Adopted Development Frameworks}: includes adopted development frameworks with references to sections in the DD. It presents adopted frameworks and programming languages with their advantages and disadvantages.

	\item \textbf{Structure of the Source Code}: explains how the source code is structured both in the front end and in the back end.

    \item \textbf{Testing}: provides the main test cases applied to the the application.
    
    \item \textbf{Installation instructions}: explains how to install and deploy the
    application.

	\item \textbf{Effort Spent}: keeps track of the time spent to complete this document. The first table	defines the hours spent as a team for taking the most important decisions, the seconds	contain the individual hours
\end{enumerate}
