\chapter{Installation instructions}
\section{Server}
In this section is illustrated the procedure in order to deploy the server on a \textit{Windows} system.\newline
The same steps can be applied, with the right adjustments, to deploy on other operating systems.

\subsection{MySQL}
Download \textit{MySQL Community Server} from the \href{https://dev.mysql.com/downloads/mysql/}{official website}.\newline
Install it, along with the \textit{MySQL workbench}, following the wizard instructions and set the \textbf{root} user credentials. Be sure to store the root password in a safe place.

By default, MySQL is launched as a service at startup.\newline
To check that MySQL is running correctly, try to connect to the server via \textit{MySQL workbench} using the credentials generated.\newline
Once connected to the database server, load the two \textit{SQL Dump} provided in the \verb|Server| archive by doing \verb|File > Run SQL Script...|\newline
You should have two new schemas: \verb|np_clup| and \verb|np_clup_test|.

\subsection{Java JDK}
Now that the database is setup, download the latest \textit{Java JDK} from the \href{https://www.oracle.com/it/java/technologies/javase-jdk15-downloads.html}{official website}.\newline
Install it following the wizard instructions and set the environment variables:
\begin{itemize}
	\item \verb|JAVA_HOME|: set \verb|<path_JDK_installation>|
	\item \verb|JDK_HOME|: set \verb|<path_JDK_installation>|
	\item \verb|CLASSPATH|: set \verb|<path_JDK_installation>\lib|
	\item \verb|PATH|: add \verb|<path_JDK_installation>\bin|
\end{itemize}
\clearpage
\subsection{TomEE}
Download the latest version of \textit{TomEE plume 8.x.x} from the \href{https://tomee.apache.org/download-ng.html}{official website}.\newline
Unzip it in a handy location and open the configuration file \verb|tomee.xml| located in the \verb|conf| folder (if it does not exist, please create it).\newline
In this file add the following lines, adjusted with your database credentials:

\lstset{
	language=XML, 
	morekeywords={tomee, Resource, role, user, tomcat-users},
	basicstyle=\scriptsize,
	showstringspaces=false,
	tabsize=5,
	alsoletter=-,
	columns=fullflexible,
}

\begin{lstlisting}
	<?xml version="1.0" encoding="UTF-8"?>
	<tomee>
		<Resource id="CLupDB" type="DataSource">
			JdbcDriver com.mysql.cj.jdbc.Driver
			JdbcUrl jdbc:mysql://localhost:3306/np_clup
			UserName xxx
			Password xxx
		</Resource>
	</tomee>
\end{lstlisting}

Edit also the \verb|tomcat-users.xml| file located in the \verb|conf| folder and add somewhere in the tag \textit{tomcat-users} the \textbf{role} and \textbf{user} lines:

\begin{lstlisting}
	<?xml version="1.0" encoding="UTF-8"?>
	<tomcat-users ...>
		...
		<role rolename="manager-gui"/>
		<role rolename="manager-script"/>
		<user username="admin" password="password" roles="manager-gui, manager-script"/>
		...
	</tomcat-users>
\end{lstlisting}

Now set the environment variable \verb|CATALINA_HOME| to the path of \textit{TomEE} folder.

Finally download the Platform Independent Java MySQL Connector from the \href{https://dev.mysql.com/downloads/connector/j/}{MySQL website}, unzip it and copy the \verb|.jar| in the \verb|lib| folder of \textit{TomEE}.

\subsection{Maven}
Download \textit{Maven} from the \href{https://maven.apache.org/download.cgi}{official website}.\newline
Unzip it in a handy location and open the configuration file \verb|settings.xml| located in the \verb|conf| folder.
Add somewhere in the tag \textit{servers}, subtag of \textit{settings}, the \textbf{server} lines:

\lstset{
	language=XML, 
	morekeywords={, settings, server, servers, id, username, password, web-app, context-param, param-name, param-value, persistence, properties, persistence-unit, property},
	basicstyle=\scriptsize,
	showstringspaces=false,
	tabsize=5,
	alsoletter=-,
	columns=fullflexible,
}

\begin{lstlisting}
	<?xml version="1.0" encoding="UTF-8"?>
	<settings ...>
		...
		<servers>
			...
			<server>
				<id>TomcatServer</id>
				<username>admin</username>
				<password>password</password>
			</server>
			...
		</servers>
		...
	</settings>
\end{lstlisting}

Now add to the environment variable \verb|PATH| the path of \textit{maven} folder.

\clearpage

\subsection{Server configuration}
Before deploying you have two configure two things in the server project.

\subsubsection{Store images folder}
To set the path of the store images edit the file \verb|web.xml| in the \verb|CLupWeb\src\main\webapp\WEB-INF| folder.
Replace the \textit{param-value} of the parameter \textit{upload.location} with the desired folder.
\begin{lstlisting}
	<?xml version="1.0" encoding="UTF-8"?>
	<web-app ...>
		<context-param>
			<param-name>upload.location</param-name>
			<param-value>/clup/uploads</param-value>
		</context-param>
	...
	</web-app>
\end{lstlisting}

The example value of \verb|/clup/uploads| corresponds to \verb|C:\clup\uploads|.\newline 
\textbf{Be sure} to put the provided example images in the specified location. You can find them in the \verb|Logo| folder located in the \verb|Server| archive.

\subsubsection{Testing database credentials} 
In order to execute the integration testing, there is the need to declare the database user credentials for the test.\newline 
Edit the file \verb|persistence.xml| in the \verb|CLupEJB\src\main\resources\META-INF| folder with your database credentials.\newline
You can use the root user or any that has all the privileges on the schema \verb|np_clup_test|.

\begin{lstlisting}
	<?xml version="1.0" encoding="UTF-8" standalone="yes"?>
	<persistence ...>
		...
		<persistence-unit name="CLupEJB-testing" transaction-type="RESOURCE_LOCAL">
			...
			<properties>
				...
				<property name="javax.persistence.jdbc.user" value="dev"/>
				<property name="javax.persistence.jdbc.password" value="password"/>
				...
			</properties>
		</persistence-unit>
	</persistence>
\end{lstlisting}


\subsection{Deploy}
In order to deploy the project to TomEE, start the server executing \verb|startup.bat| in the \verb|bin| folder of \textit{TomEE}.\newline
After that, open a terminal in \verb|Server| folder of the project and run \verb|mvn clean install| and then \verb|mvn tomcat7:deploy|.\newline
During this procedures, \textit{Maven} will also perform the \verb|test| goal.\newline
The server should now be deployed. To rerun the server, only the server startup is needed.\newline
You can find the deployed server to \href{http://localhost:8080/clup}{http://localhost:8080/clup}.

If you want to run the tests you can use \verb|mvn test|.

\clearpage

\section{Client}
\subsection{APK Installation}
To install the apps, download the APK from the repo and install them on your device.\newline
To achieve that, you will be required to allow app installs from unknown sources.\newline
\subsection{Applications configuration}
The first time that you open the app, you will be required to set the server address.\newline
Just write the address where the CLup server is running (e.g. \verb|http://192.168.1.2:8080/clup/|).
