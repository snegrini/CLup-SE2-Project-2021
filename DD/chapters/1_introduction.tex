\chapter{Introduction}

\section{Purpose}
This document contains the complete design description of the \textit{Customers Line-up} system. This includes the architectural features of the system and the details of what operations each module will perform. It also shows how the requirements and use cases detailed in the \textit{CLup, Requirement Analysis and Specification Document} will be implemented in the system.

The primary audiences of this document are the software developers and testers.

\section{Scope}
Customer Line-up (CLup) is an \textbf{easy-to-use} application which aims to settle for various queuing problems faced by supermarkets and their customers.\newline
On the one side, it allows store managers to regulate the influx of people in the building and, on the other side, it saves people from having to line up and stand outside of stores for hours on end.

Customers can enter a queue in \textit{real-time} by \textbf{taking a ticket} via different channels such as Self Service Ticketing Kiosk and a Mobile App called Customer App. During this process, the user is given an estimation of the waiting time and the \textbf{leave-at-time} (i.e. the time they need to depart from their current position to reach the store). This ticket comprehends a queue number, which identify user's position in the queue, and a QR code, which is used for the ticket validation.\newline
The validation process will be performed by a store employee using a dedicated application (Store App), which will allow them to scan QR codes and submit data to the CLup Server.

The Web App, accessible through any modern web browser, will provide store managers a dashboard from which they can monitor customers flow and check the journey map of all the clients inside the store at a given time.

The Customer App also support an \textit{advanced functionality} where a customer can \textbf{book a visit} to the store by indicating the approximate expected duration of the trip and the main categories of items they intend to buy. For long-term customers, it suggests a time inferred by the system based on an analysis of the previous visits.

This application works as a digital counterpart to the common situation where people who are in line for a service retrieve a number that gives their position in the queue. The \textit{legacy system} will be completely superseded by the application. Indeed, its effectiveness is strictly bound to the number of users who use it.

\section{Glossary}
\subsection{Definitions}
\begin{center}
	\begin{tabular}{@{}p{0.25\linewidth} p{0.71\linewidth}@{}}
		\toprule
		\textbf{Term} & \textbf{Definition}\\
		\midrule
		Customers &  Identifies the store customers.\\
		Employees & Used in this document to mean both entrance-staff and cashiers.\\
		Store Pass & General term that comprehends both tickets and bookings.\\
		Ticket & Pass generated from the system which is comprehensive of the Queue number and QR code.\\
		Booking, Reservation & Pass generated from the system as a result of the reservation process.\\
		Queue number & Identify user's position in the queue.\\
		QR code & Type of matrix barcode, used by the system for the ticket validation.\\
		System & Totality of the hardware/software applications that contribute to provide the service concerned. Also referred as CLup, Application, Platform.\\
		Legacy System & Used in this document to mean the physical ticketing system where you retrieve a ticket from a stand.\\
		\bottomrule
	\end{tabular}
\end{center}

\subsection{Acronyms}
\begin{center}
	\begin{tabular}{@{}p{0.25\linewidth} p{0.71\linewidth}@{}}
		\toprule
		\textbf{Acronyms} & \textbf{Term}\\
		\midrule
		CLup & Customers Line-up\\
		RASD & Requirements Analysis and Specification Document\\
		QR & Quick Response\\
		GPS & Global Positioning System\\
		UI & User Interface\\
		API & Application Programming Interface\\
		OS & Operating System\\
		REST & REpresentational State Transfer\\
		HTTPS & HyperText Transfer Protocol Secure\\
		JSON & JavaScript Object Notation\\	
		DB & DataBase\\
		DBMS & DataBase Management System\\
		\bottomrule
	\end{tabular}
\end{center}

\subsection{Abbreviations}
\begin{center}
	\begin{tabular}{@{}p{0.25\linewidth} p{0.71\linewidth}@{}}
		\toprule
		\textbf{Abbreviations} & \textbf{Term}\\
		\midrule
		e.g. & Exempli gratia\\
		i.e. & Id est\\
		R & Requirement\\		
		\bottomrule
	\end{tabular}
\end{center}

\section{Reference documents}
\begin{itemize}
	\item Project assignment specification document.
    \item CLup, Requirements Analysis and Specification Document.
	\item Course slides on beep.
\end{itemize}

\section{Document structure}
This document is presented as it follows:
\begin{enumerate}
	\item \textbf{Introduction}: the purpose of this document along with an overall description of the main functionalities of the system.

	\item \textbf{Architectural Design}: high level overview of how	responsibilities of the system are partitioned and assigned to subsystems. It identifies each high level subsystem and the roles assigned to them. Moreover, it describes how these subsystems collaborate with each other in order to achieve the desired functionality.

	\item \textbf{User Interface Design}: it describes the functionality of the system from the user’s perspective. It shows how the user will be able to interact with the system to complete all the expected features and the feedback information that will be displayed to the user.

	\item \textbf{Requirements Traceability}: it provides a cross-reference that traces components to the requirements contained in RASD document. A tabular format is used to show which system components satisfy each of the functional requirements from the RASD.	

    \item \textbf{Implementation, Integration and Test Plan}: an explanation of how the implementation, integration and test plan will be carried out and the technologies to be used.

	\item \textbf{Effort Spent}: keeps track of the time spent to complete this document. The first table defines the hours spent as a team for taking the most important decisions, the seconds contain the individual hours.
\end{enumerate}
